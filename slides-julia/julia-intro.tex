\documentclass{beamer}
\usepackage[utf8]{inputenc}

% Tema e estilo
\usetheme{Madrid}
\usecolortheme{default}

\title{Mini-Curso de Julia}
\author{Daniel Cassimro}
\date{\today}

\begin{document}

\frame{\titlepage}

\section{Introdução ao Julia}
\begin{frame}{Introdução ao Julia}
    \begin{itemize}
        \item O que é Julia e por que usá-la?
        \item Instalação e configuração do ambiente.
        \item Primeiros passos: variáveis, tipos de dados e operações básicas.
    \end{itemize}
\end{frame}

\begin{frame}{O que é Julia e por que usá-la?}
    \begin{itemize}
        \item 
        \item Instalação e configuração do ambiente.
        \item Primeiros passos: variáveis, tipos de dados e operações básicas.
        
        
        
        
        
    \end{itemize}
\end{frame}


\section{Estruturas de Controle}
\begin{frame}{Estruturas de Controle}
    \begin{itemize}
        \item Condicionais (if, else, elseif).
        \item Laços de repetição (for, while).
        \item Funções e escopo.
    \end{itemize}
\end{frame}

\section{Vetores e Matrizes}
\begin{frame}{Vetores e Matrizes}
    \begin{itemize}
        \item Criação e manipulação de vetores e matrizes.
        \item Operações matriciais.
        \item Indexação e fatiamento.
    \end{itemize}
\end{frame}

\section{Programação Funcional}
\begin{frame}{Programação Funcional}
    \begin{itemize}
        \item Funções anônimas.
        \item Mapeamento e redução.
        \item Compreensão de listas.
    \end{itemize}
\end{frame}

\section{Gráficos com Plots.jl}
\begin{frame}{Gráficos com Plots.jl}
    \begin{itemize}
        \item Visualização de dados.
        \item Plotagem de gráficos simples.
        \item Personalização e formatação.
    \end{itemize}
\end{frame}

\section{Pacotes e Módulos}
\begin{frame}{Pacotes e Módulos}
    \begin{itemize}
        \item Gerenciamento de pacotes com Pkg.
        \item Criação e uso de módulos.
        \item Importação de funções externas.
    \end{itemize}
\end{frame}

\section{Desempenho e Otimização}
\begin{frame}{Desempenho e Otimização}
    \begin{itemize}
        \item Dicas para escrever código eficiente.
        \item Profiling e benchmarking.
        \item Uso de macros.
    \end{itemize}
\end{frame}

\section{Aplicações Práticas}
\begin{frame}{Aplicações Práticas}
    \begin{itemize}
        \item Resolvendo problemas matemáticos.
        \item Processamento de dados.
        \item Machine learning com Flux.jl.
    \end{itemize}
\end{frame}

\section{Projeto Final}
\begin{frame}{Projeto Final}
    \begin{itemize}
        \item Desenvolvimento de um projeto prático.
        \item Implementação de um algoritmo ou análise de dados.
    \end{itemize}
\end{frame}

\section{Conclusão}
\begin{frame}{Conclusão}
    \begin{itemize}
        \item Recapitulação do que foi aprendido.
        \item Recursos adicionais e comunidade Julia.
    \end{itemize}
    
    Lembre-se de praticar e explorar mais recursos da linguagem. Julia é uma ferramenta poderosa para cientistas de dados, engenheiros e pesquisadores. Divirta-se programando! 🚀🔍📊
\end{frame}

\end{document}
